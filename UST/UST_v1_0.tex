\documentclass[a4paper,11pt]{article}
\usepackage{geometry}
\geometry{margin=1in}
\usepackage{titlesec}
\usepackage{hyperref}
\usepackage{enumitem}
\usepackage{setspace}
\setstretch{1.15}

\title{UST v1.0 --- Universal Semantic Token Model}
\author{Robert Hansen}
\date{November 2025}

\begin{document}
\maketitle

\section*{Abstract}
UST v1.0 defines a universal, domain-agnostic semantic token model for deterministic AI systems.
The model provides a typed, extensible structure for representing meaning, intent, constraints, and routing data in a machine-verifiable format.
UST enables AI runtimes and orchestration engines to operate on meaning rather than raw natural language, enabling predictable behavior, reproducible flows, and scalable multi-agent coordination.

\section{Introduction}
Modern AI systems rely primarily on natural-language prompts. While flexible, prompts lack explicit structure, type safety, and deterministic interpretation.
As multi-agent systems scale, prompt-driven workflows break or drift.

The Universal Semantic Token Model (UST) establishes a unified semantic substrate that captures:
\begin{itemize}
    \item domain-neutral meaning,
    \item system intents and directives,
    \item constraints and guarantees,
    \item routing and control metadata.
\end{itemize}

UST serves as the base layer for deterministic execution, state management, and multi-agent orchestration.

\section{Design Principles}
UST is guided by five principles:

\subsection{Universality}
The schema applies across domains without modification.

\subsection{Determinism}
Tokens support reproducible interpretation regardless of runtime context.

\subsection{Extensibility}
Families and types may be added without breaking existing systems.

\subsection{Clarity}
Each token has a single, unambiguous meaning.

\subsection{Governance Compatibility}
Tokens support validation, auditing, and external governance.

\section{Token Structure}
A UST token is defined as:

\begin{verbatim}
Token {
    id: UniqueIdentifier
    family: TokenFamily
    type: TokenType
    version: Version
    payload: TypedPayload
    constraints: ConstraintSet
    metadata: Metadata
}
\end{verbatim}

\subsection{Unique Identifier}
A stable hash across family, type, version, and payload signature.

\subsection{Token Family}
A broad semantic category, e.g., Semantic, Teleo, Trade.

\subsection{Token Type}
A subtype representing a specific semantic role.

\subsection{Version}
Semantic version of the token type.

\subsection{Typed Payload}
Structured content with type guarantees.

\subsection{Constraints}
Optional validation, safety, or behavioral limits.

\subsection{Metadata}
Contextual information including timestamps or provenance.

\section{Token Families in v1.0}
Three families are standardized:

\subsection{Semantic Tokens}
Represent descriptive information, concepts, or relationships.

\subsection{Teleo Tokens}
Represent goals, objectives, or directed intent.

\subsection{Trade Tokens}
Represent value exchange, negotiation, or commitments.

\section{Deterministic Interpretation}
Deterministic behavior emerges from:
\begin{itemize}
    \item typed payload specifications,
    \item versioned schemas,
    \item explicit constraints,
    \item stable identifiers,
    \item predictable validation rules.
\end{itemize}

Agents must not infer meaning beyond the token contents.

\section{Validation Rules}
Each token must satisfy:
\begin{enumerate}
    \item Schema validity
    \item Type-family consistency
    \item Payload-type correctness
    \item Constraint consistency
    \item Version registration
\end{enumerate}

Invalid tokens must not be interpreted.

\section{Extending UST}
Extensions require:
\begin{itemize}
    \item unique namespaces,
    \item defined payload schemas,
    \item defined validation rules,
    \item proof of determinism.
\end{itemize}

Backward-compatible additions do not require major version changes.

\section{Integration with Orchestration Engines}
UST integrates with deterministic planners and multi-agent systems.
Tokens support:
\begin{itemize}
    \item execution planning,
    \item state transitions,
    \item constraint enforcement,
    \item agent coordination,
    \item safe routing.
\end{itemize}

\section{Example Token}
\begin{verbatim}
Token {
    id: "sem_v1_concept_abc123",
    family: "Semantic",
    type: "Concept",
    version: "1.0",
    payload: {
        label: "Pipeline",
        attributes: ["Deterministic", "Auditable"]
    },
    constraints: {
        readonly: true
    },
    metadata: {
        timestamp: "2025-11-25T10:04Z"
    }
}
\end{verbatim}

\section{Future Directions}
Potential areas include:
\begin{itemize}
    \item temporal semantics,
    \item probabilistic payloads,
    \item multi-token bindings,
    \item domain expansions.
\end{itemize}

\section{Conclusion}
UST v1.0 provides a universal semantic representation suitable for deterministic AI systems, forming the semantic backbone for scalable, auditable, and reproducible multi-agent workflows.

\end{document}
