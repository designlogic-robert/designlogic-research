% =====================================================================
% Trade Tokens (TrT) v1.0
% Research Brief (2025)
% Author: Robert Hansen — Chief Semantic Architect
%
% Apache License 2.0 (hidden from PDF output)
%
% Copyright 2025 Robert Hansen
%
% Licensed under the Apache License, Version 2.0 (the "License");
% you may not use this file except in compliance with the License.
% You may obtain a copy of the License at
%
%     http://www.apache.org/licenses/LICENSE-2.0
%
% Unless required by applicable law or agreed to in writing, software
% distributed under the License is distributed on an
% "AS IS" BASIS, WITHOUT WARRANTIES OR CONDITIONS OF
% ANY KIND, either express or implied.
% =====================================================================

\documentclass[11pt]{article}
\usepackage{times}
\usepackage{amsmath, amssymb}
\usepackage{enumitem}
\usepackage{graphicx}
\usepackage{hyperref}
\usepackage{geometry}
\geometry{margin=1in}

\title{
\textbf{Trade Tokens (TrT) v1.0} \\
Semantic, Auditable Structures for Financial Intent \\
\large Research Brief — 2025
}

\author{
Robert Hansen \\
Chief Semantic Architect \\
\texttt{github.com/designlogic-robert}
}

\date{2025}

\begin{document}
\maketitle

\begin{abstract}
Trade Tokens (TrT) v1.0 define the semantic foundation for representing
trading intent, risk, position structure, and execution preferences under the
Universal Semantic Token (UST) Model. Where legacy trading systems rely on
fragile numeric pipelines, ad hoc JSON blobs, or proprietary broker formats,
Trade Tokens provide a stable, typed, engine-neutral meaning substrate for use
across human traders, agents, financial cognitive engines (FinCE), and the
Universal Semantic Runtime (USR).

This research brief formally defines the Trade Token schema, invariants,
semantics, lifecycle integration with the Trading Control Protocol (TrCP) v1.0,
and its embedding in the broader Universal Semantic Architecture (USE, USR,
UST). Trade Tokens serve as the canonical interface between semantic financial
intent and executable financial behavior across simulated, hybrid, and live
environments.
\end{abstract}

\tableofcontents
\newpage

% ---------------------------------------------------------
\section{Introduction}
Trading is one of the highest-risk domains for artificial intelligence due to
its complexity: continuous uncertainty, leverage amplification, adversarial
participants, regulatory constraints, and extreme sensitivity to execution
timing.

Most ``AI trading'' systems today rely on brittle pipelines:

\begin{itemize}
    \item indicators and raw numeric signals with no semantic meaning,
    \item opaque agent decisions without auditable reasoning,
    \item direct broker API calls without structural risk constraints,
    \item execution logic mixed with strategic logic,
    \item no global representation of trader intent.
\end{itemize}

Trade Tokens (TrT) solve this by introducing a formal, semantic structure that
represents trading intent in a stable and auditable way. They express:

\begin{itemize}
    \item what the trader or agent intends to do,
    \item under what risk and leverage constraints,
    \item with which entry and exit rules,
    \item within which strategic framework,
    \item with an explicit lifecycle traceable from idea to journaled result.
\end{itemize}

Trade Tokens do not depend on any specific trading engine, broker, or strategy.
They function as the ``semantic payload'' that the Trading Control Protocol
manages, USE executes, and USR governs.

% ---------------------------------------------------------
\section{Position of Trade Tokens in USS Architecture}

Trade Tokens live inside the Universal Semantic Token (UST) Model.

\begin{center}
\begin{verbatim}
UST (Universal Semantic Tokens)
   ├── Semantic Tokens (ST)
   ├── Teleo Tokens (TT)
   ├── Trade Tokens (TrT)
   └── Custom Domain Token Families
\end{verbatim}
\end{center}

In the financial domain:

\begin{itemize}
    \item \textbf{TrT} expresses structured trading intent.
    \item \textbf{TrCP} governs control, validation, and execution lifecycle.
    \item \textbf{FinCE} provides domain intelligence and execution modeling.
    \item \textbf{USE} handles plan decomposition and broker sequencing.
    \item \textbf{USR} enforces invariants, safety, and routing.
\end{itemize}

TrT is the declarative center of this stack. Everything above and below it
reacts to the meaning encoded in these tokens.

% ---------------------------------------------------------
\section{Design Goals}

Trade Tokens are designed around six foundational goals:

\subsection*{1. Engine-Neutral Representation}
Trading platforms, broker APIs, exchanges, and asset classes vary widely.
Trade Tokens normalize all of them into a unified, typed semantic substrate.

\subsection*{2. Auditability and Explainability}
All fields must be explicit, introspectable, and convertible to execution logs,
risk evaluations, and journaling artifacts.

\subsection*{3. Deterministic Semantics}
Two identical Trade Tokens on two different engines must express the same
intent and must lead to equivalent execution plans, subject to market
conditions.

\subsection*{4. Risk-Constrained by Construction}
Trade Tokens embed risk boundaries directly into the structure so that intent
cannot exceed declared limits.

\subsection*{5. Compatible with Human, Hybrid, and Agent Modes}
Trade Tokens must be readable and writable by humans, USR-managed agents, or
hybrid collaborations.

\subsection*{6. Broker-Independent}
No broker-specific information appears inside a Trade Token. Broker
translations are the job of FinCE and USE.

% ---------------------------------------------------------
\section{Trade Token Schema (TrT v1.0)}

Trade Tokens follow a canonical schema designed to express all meaning required
for unambiguous, risk-bounded execution.

\subsection{Top-Level Structure}

\begin{verbatim}
TRADE_TOKEN {
    ID: <uuid>,
    INSTRUMENT: <InstrumentToken>,
    POSITION: <PositionIntent>,
    RISK: <RiskModel>,
    EXECUTION: <ExecutionPrefs>,
    TARGETS: <List of ExitTargets>,
    STRATEGY: <StrategyContext>,
    TIMEFRAME: <TemporalContext>,
    META: <MetadataBlock>
}
\end{verbatim}

Each major block is defined below.

% ---------------------------------------------------------
\subsection{InstrumentToken}

\begin{verbatim}
INSTRUMENT {
    SYMBOL: "AAPL",
    VENUE: "NASDAQ",
    ASSET_CLASS: EQUITY,
    CONTRACT: OPTIONAL (for futures/options)
}
\end{verbatim}

The instrument must uniquely define the traded asset.

% ---------------------------------------------------------
\subsection{PositionIntent}

\begin{verbatim}
POSITION {
    DIRECTION: LONG | SHORT,
    SIZE: <units OR notional>,
    EXPOSURE_TYPE: UNIT | NOTIONAL | LEVERAGED
}
\end{verbatim}

% ---------------------------------------------------------
\subsection{RiskModel}

\begin{verbatim}
RISK {
    MAX_LOSS: <currency OR percent>,
    STOP: <stop_price OR algorithmic_rule>,
    PORTFOLIO_LIMIT: <float>,
    LEVERAGE_LIMIT: <float>
}
\end{verbatim}

The risk model must be internally consistent with position size and capital
constraints. USR enforces this through TrCP invariants.

% ---------------------------------------------------------
\subsection{ExecutionPrefs}

\begin{verbatim}
EXECUTION {
    ENTRY_TYPE: MARKET | LIMIT | STOP | HYBRID,
    LIMIT_PRICE: OPTIONAL,
    SLIPPAGE_MAX: <percent>,
    TIME_VALIDITY: DAY | GTC | SESSION
}
\end{verbatim}

% ---------------------------------------------------------
\subsection{ExitTargets}

\begin{verbatim}
TARGETS: [
    { PRICE: <float>, SIZE: <fraction>, CONDITIONS: [...] },
    ...
]
\end{verbatim}

Multiple scaling targets are allowed.

% ---------------------------------------------------------
\subsection{StrategyContext}

\begin{verbatim}
STRATEGY {
    LABEL: "trend_follow",
    NOTES: "Breakout play on high volume"
}
\end{verbatim}

% ---------------------------------------------------------
\subsection{TemporalContext}

\begin{verbatim}
TIMEFRAME: INTRADAY | SWING | POSITION
\end{verbatim}

% ---------------------------------------------------------
\subsection{MetadataBlock}

\begin{verbatim}
META {
    TAGS: [...],
    ORIGIN: HUMAN | AGENT | HYBRID,
    NOTES: <string>
}
\end{verbatim}

% ---------------------------------------------------------
\section{Semantic Invariants}

Trade Tokens are governed by four core invariants.

\subsection{Invariant 1: Intent–Execution Coherence}
Execution must not exceed or contradict token-defined intent.

\subsection{Invariant 2: Risk Boundary Integrity}
Risk fields must be internally consistent:

\[
\text{riskUsed} \le \text{MAX\_LOSS}
\]

\subsection{Invariant 3: Directional Fidelity}
Filled orders must match declared DIRECTION unless explicitly transformed
(e.g., hedging rules).

\subsection{Invariant 4: Semantic Round-Trip Stability}
A Trade Token must be recoverable from its execution logs, producing an
equivalent semantic structure.

% ---------------------------------------------------------
\section{Lifecycle Under Trading Control Protocol (TrCP)}

Trade Tokens move through the following phases in TrCP:

\begin{enumerate}
    \item DRAFT — initial construction
    \item READY — schema-valid and semantically consistent
    \item RISK\_CHECKED — evaluated by TrCP's risk engine
    \item AUTHORIZED — explicit human or policy greenlight
    \item EXECUTING — orders hit the market
    \item ACTIVE — position is live
    \item CLOSED — position resolved
    \item JOURNALED — full round-trip integration
\end{enumerate}

Trade Tokens define the structure; TrCP governs the movement.

% ---------------------------------------------------------
\section{Integration with USR}

USR handles:

\begin{itemize}
    \item validation of TrT schema and fields,
    \item routing through TrCP,
    \item enforcement of semantic invariants,
    \item logging semantic events,
    \item multi-engine orchestration between FinCE, USE, and analysis modules.
\end{itemize}

USR sees a Trade Token as a semantic contract: a promise about intent that must
not be violated.

% ---------------------------------------------------------
\section{Integration with USE}

USE transforms Trade Tokens into execution plans:

\begin{itemize}
    \item decomposing intent into atomic buy/sell actions,
    \item applying broker-neutral translation,
    \item sequencing adjustments and conditional logic,
    \item generating semantic deltas after fills.
\end{itemize}

Trade Tokens provide the ``what.''  
USE provides the ``how.''

% ---------------------------------------------------------
\section{Integration with FinCE}

FinCE is the domain engine for finance:

\begin{itemize}
    \item produces suggested trade structures,
    \item evaluates setups and edge,
    \item conducts backtests and forward tests,
    \item provides continuous analysis for active positions.
\end{itemize}

FinCE emits and consumes Trade Tokens as its primary semantic interface.

% ---------------------------------------------------------
\section{Examples}

Two simplified examples:

\subsection*{Example 1: Human Hybrid Intent}

\begin{verbatim}
AAPL long breakout
Entry: limit 182.40
Stop: 178.90
Target: 190.00
Risk: max 1% of capital
\end{verbatim}

This becomes a structured TrT with explicit fields and invariants.

\subsection*{Example 2: Agent-Generated Setup}

Agent detects high-volume reversal on ETHUSD:

\begin{itemize}
    \item SHORT entry at 3520
    \item Stop at 3600
    \item Two targets
    \item Max exposure: 0.5% equity
\end{itemize}

Agent proposes the TrT → TrCP evaluates → USE executes if approved.

% ---------------------------------------------------------
\section{Conclusion}

Trade Tokens v1.0 provide a deterministic semantic substrate for trading intent,
risk, and execution structure. They unify human and agent workflows under USR,
enable safe execution under TrCP, and integrate cleanly with FinCE and USE.

They are the meaning backbone of a semantic financial engine.

\end{document}

