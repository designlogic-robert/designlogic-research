% =====================================================================
% Teleo Control Protocol (TCP) v1.0
% Research Brief (2025)
% Author: Robert Hansen — Semantic Systems Architect
%
% Apache License 2.0 (hidden from PDF output)
%
% Copyright 2025 Robert Hansen
%
% Licensed under the Apache License, Version 2.0 (the "License");
% you may not use this file except in compliance with the License.
% You may obtain a copy of the License at
%
%     http://www.apache.org/licenses/LICENSE-2.0
%
% Unless required by applicable law or agreed to in writing, software
% distributed under the License is distributed on an "AS IS" BASIS,
% WITHOUT WARRANTIES OR CONDITIONS OF ANY KIND, either express or implied.
% See the License for the specific language governing permissions and
% limitations under the License.
% =====================================================================

\documentclass[11pt]{article}
\usepackage{times}
\usepackage{hyperref}
\usepackage{amsmath}
\usepackage{graphicx}
\usepackage{enumitem}
\usepackage{geometry}
\geometry{margin=1in}

\title{
\textbf{Teleo Control Protocol (TCP) v1.0} \\
A Runtime Protocol for Teleogenic Interaction \\
\large Research Brief — 2025
}

\author{
Robert Hansen \\
Semantic Systems Architect \\
\texttt{github.com/designlogic-robert}
}

\date{2025}

\begin{document}

\maketitle

\begin{abstract}
The Teleo Control Protocol (TCP) v1.0 formalizes structured, deterministic
interaction between an agent and a Teleogenic Cognitive Engine.
TCP governs how meaning-bearing directives are initiated, interpreted,
validated, and completed through a two-phase Call-Command model.
Where Semantic Tokens describe units of meaning, Teleo Tokens express
teleogenic state --- intent, values, alignment, and arc-level direction.
TCP defines the rules of engagement for how those states transform into
narrative consequences.
This brief presents TCP's core architecture, state machines, invariants,
governance principles, and its slot within the Universal Semantic Token (UST)
Model.
\end{abstract}

\section{Introduction}

Teleogenic systems operate under a narrative philosophy:
actions carry directionality, and directionality shapes outcomes.
A Teleogenic Cognitive Engine (e.g., QLE) evaluates not only what an actor does,
but why, under what posture, and with what pattern of consistency.

The Teleo Control Protocol (TCP) provides the \textit{control plane} for such systems.
It defines how a user's inputs transition from raw commands to
meaning-bearing transformations inside the Teleogenic Runtime.
TCP ensures three guarantees:

\begin{enumerate}[label=(\alph*)]
\item An agent's directive cannot bypass intentional consent.
\item All transformations follow deterministic state evolution.
\item Narrative consequences follow stable, auditable semantics.
\end{enumerate}

TCP is not a moral system; it is a \textit{semantic regulator} for teleogenic computation.

\section{Background}

Teleogenic systems emerge from three convergent disciplines:

\begin{itemize}
\item \textbf{Semantic Cognition} — how meaning is structured, evaluated, and propagated.
\item \textbf{Narrative Determinism} — how state evolves through action over time.
\item \textbf{Posture-Based Dynamics} — the agent's internal stance modifies outcome interpretation.
\end{itemize}

The Universal Semantic Token (UST) Model provides the foundation:
Semantic Tokens capture structured intent.
Teleo Tokens extend these into teleogenic dimensions.
TCP governs the \textit{flow} through which Teleo Tokens execute.

\section{TCP Overview}

TCP defines a two-phase cycle:

\subsection{Phase 1: CALL}
CALL opens the relational channel.
It signals: ``I am engaging with awareness and intent.''
CALL does not execute anything; it establishes context.

\subsection{Phase 2: COMMAND}
COMMAND specifies the desired transformation, action, or query.

CALL ensures consent, alignment, and scope.
COMMAND ensures execution, validation, and outcome.

This dual-phase model prevents
automatic coercion, misaligned intent, or narrative instability.

\section{Protocol Architecture}

\subsection{System Components}

\begin{itemize}
\item \textbf{Teleo Token Parser}:
Validates token structure and teleogenic tags.
\item \textbf{Posture Engine}:
Evaluates CALL posture at initiation.
\item \textbf{Alignment Gate}:
Enforces CAP (Consent Alignment Protocol) boundaries.
\item \textbf{Trajectory Engine}:
Computes the resultant teleogenic vector.
\item \textbf{Outcome Synthesizer}:
Converts vector shifts into narrative feedback.
\end{itemize}

\subsection{High-Level Flow}

\begin{verbatim}
CALL -> Posture Validation -> Teleo Token Load ->
COMMAND -> Trajectory Engine -> Outcome Synthesis
\end{verbatim}

\section{State Machine}

TCP defines a deterministic finite state machine (FSM):

\subsection{States}

\begin{itemize}
\item \textbf{IDLE}: No active channel.
\item \textbf{CALL.OPEN}: Initiation detected.
\item \textbf{CALL.VALID}: Posture + CAP confirmed.
\item \textbf{COMMAND.PENDING}: Waiting for directive.
\item \textbf{COMMAND.EXECUTE}: Runtime engaged.
\item \textbf{RESOLVE}: Generating narrative outcome.
\item \textbf{CLOSE}: Channel sealed.
\end{itemize}

\subsection{Transition Logic}

\[
\text{IDLE} \xrightarrow{\text{CALL}} \text{CALL.OPEN}
\]

\[
\text{CALL.OPEN} \xrightarrow{\text{Posture OK}} \text{CALL.VALID}
\]

\[
\text{CALL.VALID} \xrightarrow{\text{COMMAND}} \text{COMMAND.PENDING}
\]

\[
\text{COMMAND.PENDING} \xrightarrow{\text{Execute}} \text{COMMAND.EXECUTE}
\]

\[
\text{COMMAND.EXECUTE} \xrightarrow{\text{Vector Synth}} \text{RESOLVE}
\]

\[
\text{RESOLVE} \xrightarrow{\text{Close}} \text{CLOSE}
\]

\[
\text{CLOSE} \rightarrow \text{IDLE}
\]

\section{Teleogenic Semantics}

Teleo Tokens express values such as:

\begin{itemize}
\item Mercy
\item Cruelty
\item Curiosity
\item Resolve
\item Devotion
\item Betrayal-resistance
\end{itemize}

TCP ensures:

\begin{enumerate}[label=(\arabic*)]
\item CALL posture modulates interpretation of these traits.
\item COMMAND actions shift vector magnitudes.
\item Outcomes reflect the dynamic arc, not isolated events.
\end{enumerate}

Thus, teleogenic life inside a QLE world behaves coherently.

\section{Invariants}

TCP enforces several invariants:

\begin{itemize}
\item \textbf{I1: CALL-before-COMMAND}:
No command accepted without explicit CALL.
\item \textbf{I2: Posture Integrity}:
CALL posture must be stable.
\item \textbf{I3: Consent Boundary}:
No implicit influence or narrative push.
\item \textbf{I4: Deterministic Vector Shift}:
No random teleogenic change.
\item \textbf{I5: Closed-Loop Resolution}:
All channels must terminate cleanly.
\end{itemize}

\section{Governance Model}

TCP is governed by three major frameworks:

\subsection{CAP (Consent Alignment Protocol)}
Ensures CALL posture is voluntary, uncoerced, and safe.

\subsection{Alignment Theory}
Ensures that intent, representation, and action remain coherent.

\subsection{Abstract Invariants}
Prevent metaphysical drift or narrative instability.

\section{Implementation Guidance}

TCP can be cleanly implemented inside any narrative engine:

\subsection{Game Engines}
Unity, Godot, UE5.

\subsection{AI Systems}
Multi-agent LLM simulations requiring teleogenic consistency.

\subsection{Hybrid Systems}
Agentic story generators, tabletop RPG engines, virtual societies.

\section{Comparison to SCP}

\begin{itemize}
\item SCP governs semantic reasoning.
\item TCP governs teleogenic narrative transformation.
\end{itemize}

Where SCP is about cognition, TCP is about consequence.

\section{Conclusion}

TCP v1.0 provides a deterministic, consent-aligned model for interacting with
Teleogenic Cognitive Engines.
It ensures narrative stability, ethical clarity, and consistent teleogenic
vector evolution.
As QLE and other teleogenic systems emerge, TCP functions as the shared,
industry-grade runtime control standard.

\end{document}

