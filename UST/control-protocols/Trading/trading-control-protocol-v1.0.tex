% =====================================================================
% Trading Control Protocol (TrCP) v1.0
% Research Brief (2025)
% Author: Robert Hansen — Chief Semantic Architect
%
% Apache License 2.0 (hidden from PDF output)
%
% Copyright 2025 Robert Hansen
%
% Licensed under the Apache License, Version 2.0 (the "License");
% you may not use this file except in compliance with the License.
% You may obtain a copy of the License at
%
%     http://www.apache.org/licenses/LICENSE-2.0
%
% Unless required by applicable law or agreed to in writing, software
% distributed under the License is distributed on an "AS IS" BASIS,
% WITHOUT WARRANTIES OR CONDITIONS OF ANY KIND, either express or implied.
% See the License for the specific language governing permissions and
% limitations under the License.
% =====================================================================

\documentclass[11pt]{article}
\usepackage{times}
\usepackage{hyperref}
\usepackage{amsmath, amssymb}
\usepackage{graphicx}
\usepackage{enumitem}
\usepackage{geometry}
\geometry{margin=1in}

\title{
\textbf{Trading Control Protocol (TrCP) v1.0} \\
A Runtime Control Layer for TradeToken-Governed Markets \\
\large Research Brief — 2025
}

\author{
Robert Hansen \\
Chief Semantic Architect \\
\texttt{github.com/designlogic-robert}
}

\date{2025}

\begin{document}

\maketitle

\begin{abstract}
The Trading Control Protocol (TrCP) v1.0 defines a deterministic, safety-aware
control layer for executing trading intentions under the Universal Semantic
Token (UST) Model using Trade Tokens (TrT). Where Trade Tokens encode the
semantic structure of positions, risk, and intent, TrCP governs the operational
lifecycle of a trade: from initial hypothesis to order placement, monitoring,
adjustment, and closure. This research brief presents the protocol's state
model, invariants, governance principles, and integration points with the
Universal Semantic Runtime (USR), the Universal Semantic Engine (USE), and
domain-specific trading engines (FinCE). TrCP is designed to be engine-agnostic,
broker-neutral, and auditable, making it suitable for both algorithmic and
assisted human trading systems.
\end{abstract}

\tableofcontents
\newpage

\section{Introduction}

Trading is an unusually dense domain: information, risk, leverage, latency,
and psychology collide. Traditional algorithmic trading frameworks operate in
a numerical mode --- prices, indicators, signals --- with only weak structural
representations of trader intent or risk posture.

The Trading Control Protocol (TrCP) v1.0 addresses this gap by:

\begin{itemize}
    \item formalizing the lifecycle of a trade as a semantic process
    \item ensuring that TradeToken-governed intent is honored without being exceeded
    \item enforcing invariants around risk, leverage, and compliance
    \item enabling transparent, auditable flows from idea to executed position
\end{itemize}

TrCP does not decide \emph{what} to trade. Instead, it defines \emph{how}
trade instructions, once encoded as Trade Tokens, may safely and coherently
flow through a runtime, from human or agent intent into concrete orders and
back into semantic state.

\section{Background}

\subsection{Universal Semantic Token Model and Trade Tokens}

The Universal Semantic Token (UST) Model provides a typed substrate for
representing structured meaning. Within that model, Trade Tokens (TrT) form a
specialized family for describing:

\begin{itemize}
    \item instruments (e.g., NASDAQ equities, futures, options)
    \item position structures (entry, size, stop, target, timeframe)
    \item risk constraints (max loss, max exposure, leverage rules)
    \item strategy context (trend-following, mean reversion, breakout)
    \item execution preferences (limit, market, partial fills, slippage bounds)
\end{itemize}

Trade Tokens are designed to be independent of specific broker APIs or
platforms. They express \emph{what} the trader or agent wants in a stable,
schema-governed form.

\subsection{The Role of TrCP}

If Trade Tokens define the ``semantic payload'' of a trade,
TrCP defines the \emph{control protocol} by which:

\begin{enumerate}[label=(\alph*)]
    \item those tokens are validated and risk-checked,
    \item transformed into one or more execution plans,
    \item sequenced into broker-level orders,
    \item monitored, adjusted, and closed, and
    \item reintegrated into TradeToken-level state for journaling and learning.
\end{enumerate}

TrCP lives between semantic intent (UST/TrT) and execution environments
(FinCE, broker APIs, backtesting engines, simulation sandboxes).

\section{Protocol Overview}

\subsection{High-Level Phases}

TrCP is organized into six canonical phases:

\begin{enumerate}[label=\textbf{P\arabic*:}]
    \item \textbf{PREPARE} — ingest, validate, and normalize Trade Tokens.
    \item \textbf{EVALUATE} — risk, edge, and constraint evaluation.
    \item \textbf{AUTHORIZE} — explicit approval under risk and governance rules.
    \item \textbf{EXECUTE} — order generation and broker-level submission.
    \item \textbf{MONITOR} — real-time tracking, adjustments, and conditional logic.
    \item \textbf{SETTLE} — closure, journaling, and semantic reintegration.
\end{enumerate}

Each phase has clear entry and exit conditions and must satisfy specific
invariants to prevent uncontrolled risk or semantic drift.

\subsection{Conceptual Diagram}

Conceptually, the flow can be expressed as:

\begin{verbatim}
TradeTokenSet
   -> PREPARE
   -> EVALUATE (risk, edge, constraints)
   -> AUTHORIZE (explicit or rule-based)
   -> EXECUTE (orders via FinCE/Broker)
   -> MONITOR (runtime adjustments)
   -> SETTLE (results -> TradeTokenJournal)
\end{verbatim}

\section{TradeToken Structures under TrCP}

For reference, we assume a canonical Trade Token structure:

\begin{verbatim}
TRADE_TOKEN {
    INSTRUMENT: <symbol, venue, asset_class>,
    DIRECTION: LONG | SHORT,
    SIZE: <units or notional>,
    ENTRY: {
        TYPE: LIMIT | MARKET | STOP | OTHER,
        PRICE: <float or formula>,
        VALIDITY: GTC | DAY | SESSION
    },
    RISK: {
        MAX_LOSS: <currency or %>,
        STOP: <price or dynamic rule>,
        MAX_PORTFOLIO_RISK: <fraction>
    },
    TARGETS: [
        { PRICE: <float>, SIZE: <fraction> },
        ...
    ],
    TIMEFRAME: INTRADAY | SWING | POSITION,
    STRATEGY: <label or reference>,
    META: {
        TAGS: [...],
        NOTES: <string>,
        ORIGIN: HUMAN | AGENT | HYBRID
    }
}
\end{verbatim}

TrCP does not dictate the exact schema, but assumes that Trade Tokens expose at
least: instrument, direction, size, risk, and execution preferences.

\section{State Machine}

\subsection{States}

TrCP defines a finite state machine that governs the lifecycle of a
TradeTokenSet:

\begin{itemize}
    \item \textbf{S0: DRAFT} — Trade Tokens created but unvalidated.
    \item \textbf{S1: READY} — Tokens syntactically and semantically valid.
    \item \textbf{S2: RISK\_CHECKED} — Risk and constraints evaluated.
    \item \textbf{S3: AUTHORIZED} — Explicit or policy-based approval granted.
    \item \textbf{S4: EXECUTING} — Orders live in the market.
    \item \textbf{S5: ACTIVE\_POSITION} — Position exists; monitoring ongoing.
    \item \textbf{S6: CLOSED} — Position fully closed; P\&L realized.
    \item \textbf{S7: JOURNALED} — Trade integrated into semantic history.
\end{itemize}

\subsection{Transitions}

\[
\text{S0 (DRAFT)} \xrightarrow{\text{Validate}} \text{S1 (READY)}
\]

\[
\text{S1 (READY)} \xrightarrow{\text{Risk Engine OK}} \text{S2 (RISK\_CHECKED)}
\]

\[
\text{S2 (RISK\_CHECKED)} \xrightarrow{\text{User/Policy Approval}} \text{S3 (AUTHORIZED)}
\]

\[
\text{S3 (AUTHORIZED)} \xrightarrow{\text{Order Sent}} \text{S4 (EXECUTING)}
\]

\[
\text{S4 (EXECUTING)} \xrightarrow{\text{Order Filled}} \text{S5 (ACTIVE\_POSITION)}
\]

\[
\text{S5 (ACTIVE\_POSITION)} \xrightarrow{\text{Exit Conditions Met}} \text{S6 (CLOSED)}
\]

\[
\text{S6 (CLOSED)} \xrightarrow{\text{Journal Commit}} \text{S7 (JOURNALED)}
\]

Rejections, cancellations, or risk violations cause transitions to fallback or
terminal error states (not detailed here but recommended in implementations).

\section{Invariants}

TrCP enforces a set of invariants to prevent misuse and uncontrolled risk.

\subsection{I1: No Execution without Authorization}

No trade may move into EXECUTING state without passing through:

\[
\text{READY} \rightarrow \text{RISK\_CHECKED} \rightarrow \text{AUTHORIZED}
\]

Any bypass of this path is an invalid protocol violation.

\subsection{I2: Risk Budget Integrity}

Let $R_{max}$ be the configured maximum portfolio risk
and $R_{new}$ the incremental risk of the proposed trade.
Let $R_{current}$ be the risk of existing open positions.

TrCP requires:

\[
R_{current} + R_{new} \leq R_{max}
\]

If this inequality fails, the trade cannot be authorized unless other
positions are reduced or closed.

\subsection{I3: Leverage and Instrument Constraints}

Each instrument class has associated leverage and regulatory bounds.
TrCP must ensure that the effective leverage implied by SIZE, STOP, and
account capital does not exceed configured or legal maxima.

\subsection{I4: Semantic Consistency}

The TradeTokenSet must remain semantically consistent before and after
execution. For example:

\begin{itemize}
    \item Position direction must match net executed exposure.
    \item Risk fields must accurately reflect actual stops and targets.
    \item Timeframe tags must not contradict holding behavior.
\end{itemize}

\subsection{I5: Journaling Completeness}

Every finalized trade must eventually reach JOURNALED state. Trades that close
without journaling are protocol violations and should trigger alerts.

\section{Integration with USR and USE}

\subsection{USR (Universal Semantic Runtime)}

USR provides the semantic operating environment in which Trade Tokens and TrCP
live. For TrCP, USR is responsible for:

\begin{itemize}
    \item validating token types and field schemas
    \item enforcing the invariants specified above
    \item routing TrCP operations to appropriate engines (FinCE, backtester)
    \item logging semantic events for audit and learning
\end{itemize}

TrCP is implemented as a protocol module within USR, with well-defined input
(TradeTokenSet plus context) and output (state transitions plus execution
instructions).

\subsection{USE (Universal Semantic Engine)}

USE operationalizes semantic structures into concrete actions. In the context
of TrCP, USE:

\begin{itemize}
    \item transforms validated Trade Tokens into execution plans
    \item decomposes plans into atomic broker operations (orders, modifies, cancels)
    \item sequences monitoring rules and conditional adjustments
    \item generates semantic deltas (e.g., risk used, realized P\&L) back into UST
\end{itemize}

TrCP relies on USE for the ``how'' of execution, while TrCP itself governs the
``whether'' and ``when''.

\section{FinCE and Broker Adapters}

\subsection{FinCE as Trading Engine}

FinCE, as a dedicated cognitive engine for financial markets, is a natural
host for TrCP. Inside FinCE:

\begin{itemize}
    \item Trade Tokens express structured trading intent.
    \item TrCP manages the lifecycle of those tokens.
    \item Broker adapters map execution plans to specific broker APIs.
\end{itemize}

\subsection{Broker Neutrality}

TrCP is explicitly broker-neutral. All broker-specific concerns (order syntax,
API quirks, rate limits) are encapsulated in adapter layers that:

\begin{itemize}
    \item translate USE-level execution primitives into broker calls
    \item normalize broker responses into semantic events
\end{itemize}

This allows TrCP and Trade Tokens to remain stable even if execution venues
change.

\section{Failure Modes and Safeguards}

\subsection{FM1: Partial Execution}

Orders may fill partially or across multiple venues. TrCP must treat the
position as ACTIVE\_POSITION only to the extent of filled size and must
preserve invariant consistency with the fractional exposure.

\subsection{FM2: Slippage and Gaps}

Markets may gap, causing stop orders to fill worse than expected.
TrCP should:

\begin{itemize}
    \item record actual fill prices and update realized risk
    \item trigger post-trade risk review if slippage exceeds thresholds
\end{itemize}

\subsection{FM3: Connectivity Loss}

If connectivity to broker or data feeds is lost, TrCP should:

\begin{itemize}
    \item freeze new AUTHORIZE \textrightarrow EXECUTE transitions
    \item flag existing EXECUTING/ACTIVE\_POSITION states for manual review
\end{itemize}

\subsection{FM4: Protocol Violations}

Any attempt to bypass phases, ignore risk constraints, or execute trades
without journaling should be logged as a critical protocol violation, with
appropriate alerts to operators or supervisory systems.

\section{Human-in-the-Loop vs Fully Automated Modes}

TrCP is designed to support a spectrum of autonomy:

\subsection{Assisted Mode}

\begin{itemize}
    \item Humans specify Trade Tokens.
    \item TrCP performs risk checks and recommends actions.
    \item Humans explicitly approve AUTHORIZE transitions.
\end{itemize}

\subsection{Hybrid Mode}

\begin{itemize}
    \item Agents propose Trade Tokens under policy bounds.
    \item Humans review batches or thresholds.
\end{itemize}

\subsection{Automated Mode}

\begin{itemize}
    \item Agents generate and authorize trades under strict policy constraints.
    \item TrCP still enforces invariants and journaling.
\end{itemize}

In all cases, TrCP ensures that the same state machine and invariants apply.

\section{Example Flow}

A simplified example:

\begin{enumerate}
    \item A strategy engine proposes a LONG trade on a NASDAQ stock, with defined stop and target.
    \item Trade Token is created (S0: DRAFT).
    \item USR validates schema and semantics (S1: READY).
    \item Risk engine confirms total portfolio risk remains within bounds (S2: RISK\_CHECKED).
    \item The trader reviews and approves the trade (S3: AUTHORIZED).
    \item USE generates limit orders and submits via FinCE to the broker (S4: EXECUTING).
    \item Orders fill, position becomes live (S5: ACTIVE\_POSITION).
    \item Price hits target, exit order executes; position closes (S6: CLOSED).
    \item Trade is encoded into a journal entry and appended to TradeToken history (S7: JOURNALED).
\end{enumerate}

This flow demonstrates how semantic intent, risk integrity, and execution
interlock under TrCP.

\section{Relationship to Other Protocols}

\subsection{SCP vs TrCP}

The Semantic Control Protocol (SCP) governs semantic reasoning more broadly:
question answering, planning, multi-step workflows across arbitrary domains.
TrCP specializes in a single high-risk vertical: trading.

\begin{itemize}
    \item SCP asks: ``Which sequence of semantic operations should we run?''
    \item TrCP asks: ``Which of these trades, if any, should be allowed into the market, and under what constraints?''
\end{itemize}

\subsection{Teleo Control Protocol vs TrCP}

The Teleo Control Protocol (TCP) focuses on teleogenic arcs in narrative or
psychological contexts. TrCP can be viewed as its financial sibling: it
applies similar control-layer ideas to capital, risk, and markets.

\section{Conclusion}

The Trading Control Protocol (TrCP) v1.0 provides a structured, invariant-driven
control layer for TradeToken-based trading systems. By formalizing the trade
lifecycle as a semantic state machine --- from DRAFT to JOURNALED ---
and integrating tightly with USR, USE, and FinCE, TrCP offers:

\begin{itemize}
    \item broker-neutral, engine-agnostic trading control,
    \item rigorously bounded risk and leverage,
    \item transparent, auditable execution flows,
    \item compatibility with human, hybrid, and automated operation.
\end{itemize}

As semantic trading architectures mature, TrCP can serve as a foundational
runtime standard for safe, interpretable, and semantically coherent market
interaction.

\end{document}

